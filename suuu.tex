\documentclass[12pt]{article}

\usepackage{graphicx}
\graphicspath{{downloads/}}


\begin{document}



\section{Acknowledgement}
At the onset of this very report , I'd want to express my honest and emotional gratitude to everyone who has assisted me in this attempt. I would not have progressed in the project without their active advice, assistance, cooperation, and encouragement.
                                                                 I'd like to express my heartfelt gratitude to my teacher, Mr. Saurabh Gupta, for his guidance and support, as well as this College, NIT Raipur, for providing me with the wonderful opportunity to work on this wonderful project on the topic "Augmented Reality in Health Care," which allowed me to do a lot of research and learn a lot of new things.
I owe them a debt of gratitude.
Second, I'd like to express my gratitude to my parents and friends who helped me a lot In completing this topic in a limited amount of time.

\paragraph{~}
Thanks
\linebreak
Surbhi Kosare
\pagebreak
\section{Abstract}
In this project, I'll discuss how Augmented Reality is being used in the medical field. The use of augmented reality has become significantly widespread  over the last decade. Professionals in healthcare are using augmented reality-based tools and features to communicate, educate, and engage patients, as well as to develop new inventive ways for a variety of other objectives like Medical training, surgery preparation, better symptoms description, better drug information etc. To better understand the impact of the invention, we'll look at different areas of medical care where it's being used, map the impact against the previous situation, and calculate the worth of the innovation.
\pagebreak
\section{Augmented Reality in healthcare}

\subsection{Introduction}
Augmented reality is a view of the real, physical world in which elements are enhanced by computer-generated input. These inputs may range from sound to video, to graphics to GPS overlays and more. The first conception of augmented reality occurred in a novel by Frank L Baum written in 1901 in which a set of electronic glasses mapped data onto people; it was called a “character marker”. Today, augmented reality is a real thing and not a science-fiction concept.
Now the question is how really Augmented Reality works?
                                                 AR works by "[adding] digital content onto a live camera feed, making that digital content look as if it is part of the physical world around you."
This is generally achieved by using computer vision, which is a trait that differentiates AR from VR, which transports users into completely digital worlds using headsets and sometimes haptic sensors.


                                                            
                                                            
                                                            Virtual Reality (VR) is frequently confused with Augmented Reality (AR) (VR). The fundamental distinction between the two is that while Virtual Reality replaces the entire real environment with an artificial one, Augmented Reality applies features such as audio, films, and graphics to an existing real environment.
In actual world, with the purpose of being instructive and offering additional data about the real world that a user can access without having to look for it. When a handset is In augmented reality (AR), a virtual environment is created to cohabit with the pointed at a piece of malfunctioning equipment, for example, industrial AR apps might provide rapid troubleshooting information.
Virtual reality is a comprehensive environmental simulation in which the user's real world is replaced with a completely virtual one. These virtual settings are frequently created to be larger than life because they are fully synthetic. In a virtual boxing ring, for example, a user may fight a cartoon version of Mike Tyson.
            Both virtual reality and augmented reality are intended to provide a simulated experience.
            
            \clearpage
\subsection{Uses and Innovation prospects}
Augmented reality is now used in medical training. Its applications range from MRI applications to performing highly delicate surgery
At the Cleveland Clinic at Case Western Reserve University, for example, students are taught the ins and outs of anatomy using AR headsets or augmented reality glasses. This technology lets them delve into the human body without the need for dissecting cadavers or watching live operations.
•	AR also has applications during operations where it can help reduce the need for more traditional invasive cameras and probes.
InnerOptic Technology's Magic Loupe, for example, integrates with Osterhout Design Group (ODG) and Microsoft's HoloLens to improve the doctor's view of the patient's insides.
This has the potential to make invasive surgeries more precise and safe. 
Below are some of the applications of augmented reality (AR) for the industry:
\subsubsection{	Medical imaging assists surgeons in OR}
Doctors and even patients understand the need of precision when it comes to surgery. AR can now assist surgeons in being more efficient during operations. AR healthcare apps can help save lives and treat patients effortlessly, whether they're performing a minimally invasive treatment or identifying a tumor in the liver.

Sync AR created software that gives surgeons "X-ray vision" by integrating digitally augmented photos directly into a surgical device's microscope.

AR enhances visualization of CT or MRI data by superimposing stereoscopic projections during a surgical procedure. This information is vital in surgeries requiring precise navigation to a particular organ. For example, AR can be used for pre-operative planning enhances accurate localization of tumors and surrounding structures for performing procedures such as minimally invasive partial nephrectomy or radical prostatectomy where the challenging anatomy of the vascular or nervous system could complicate the tumor removal.

\begin{figure}[h]
\centering
\includegraphics[scale=0.6]{ar4}
\end{figure}
\subsubsection{ Helping the patients by better symptoms recognition}
When it comes to appropriately describing their symptoms to their doctors, patients frequently stumble. Other times, people find themselves overreacting to a medical crisis – or, on the contrary, dismissing the issue. Augmented reality in ophthalmology could be the answer to patient education.

Oculenz is a treatment for people who have lost their centre vision. Doctors can simulate the eyesight of a patient suffering from a specific ailment using apps like Oculenz. Patients may be more motivated to make positive changes if they can see the long-term effects of their lifestyle on their health.
                                       VA-ST’s SmartSpecs enhances the visual appearance of everyday objects and people using 3D recognition software. It h=elps legally blind people or those with serious visual impairments recognize familiar faces, find lost items, and easily navigate their environment.
                                       
\begin{figure}[h]
\centering
\includegraphics[scale=0.6]{ar}
\end{figure}                                       
\subsubsection{Hospital Navigation}
Modern hospitals are complex facilities that both patients and new staff can find difficult to navigate. One useful AR function is designing navigation and way finding tools that help anyone orient themselves to the environment. When made accessible through a smartphone app, these tools are valuable to newcomers and relieve pressure on support staff. 
                                 Apart from traditional navigation, augmented reality has the ability to provide real-time aid in emergency situations. In the event of a fire, earthquake, or other disaster, hospital employees may be required to evacuate people as soon as possible. A well-designed augmented reality navigation technology can lead people to exits on paths that avoid danger zones and avoid crowding.

\subsubsection{Remote surgical expertise}
VIPAR (Virtual Interactive Presence in Augmented Reality) is a video support solution that surpasses telemedicine. A surgeon remotely guides a peer during a procedure by projecting his hands into an AR display. 
                                   During the procedure, VIPAR enabled both surgeons to communicate in a complicated visual and spoken manner. In comparison to the audio signal, 5 video clips showed a visual delay of 237 milliseconds (range, 93391 milliseconds). The remote neurosurgeon was able to see all of the important anatomy due to the high imaging resolution. The remote neurosurgeon could gesture at structures with submillimeter precision, with no discernible difference in accuracy between stations. Between Vietnam and the United States, fifteen endoscopic third ventriculostomy with choroid plexus coagulation procedures have been conducted with VIPAR, with no severe problems. Eighty percent of these patients do not require a shunt.
                                                                  Long-distance, intraoperative guiding, and knowledge transmission technologies are evolving, and they have a lot of potential for extremely efficient international neurosurgery education. One example of a low-cost, scalable platform for expanding worldwide neurosurgical capacity is VIPAR. Efforts are being made to form a network of Vietnamese neurosurgeons who will collaborate using VIPAR.
                                                                  

\begin{figure}[h]
\centering
\includegraphics[scale=0.5]{ar2}

\end{figure}                                                                  
\subsubsection{	Augmnented Reality showing defibrillators and saves lives }
Consider the following scenario: you're with a relative who suddenly passes out. What would you do in this situation? Yes, after you've calmed down from the flood of emotions, you'll want to talk to someone. You might consider contacting a doctor, an ambulance, or your friends for assistance.
Here's something to think about. In addition to the essential emergency numbers, you might consider downloading the Layar reality browser along with the AED4EU app to your phone. If you find yourself in a similar situation, you will be able to assist more people.
The AED4EU was established by Lucien Engelen, who works at the Radboud University Nijmegen Medical Centre. It can be used to add locations where automated external defibrillators (AEDs) are placed. The best part? Physicists can now use the new application to view the database.
With the Layar browser, you can also project the actual location of the nearest AEDs on your phone's screen. Then it would only take a few minutes to locate them and assist those in need.






\begin{figure}[h]
\centering
\includegraphics[scale=0.6]{ar3}

\end{figure}

\pagebreak
                                  
\subsection{Conclusion}
 With the use of augmented reality in healthcare, the industry has taken bold and technologically sophisticated leaps. It has aided and assisted in making the process easier and more convenient for both medical professionals and patients. They can also be more informed and interactive about the medications they're taking, their health conditions, and even their progress.
The healthcare system becomes more transparent, educational, and adaptable as a result of this. Augmented reality has opened up various paths for deployment in healthcare, ranging from extensive instruction of new professionals to practise of common medical operations with less space for error.
                                            Also, because it performs such important functions, it is very cost effective. Traditional medical learning tools are expensive, and understanding one's condition often necessitates a lot of medical exposure. Augmented reality works by providing affordable and descriptive solutions and alternatives in the healthcare industry.
                                            
\pagebreak                                            
\section{Refrences}
I.	https://www.resonai.com/blog/5-examples-of-augmented-reality-in-modern-healthcare-facilities
\linebreak
II.	https://medicalfuturist.com/augmented-reality-in-healthcare-will-be-revolutionary/
\linebreak
III.	https://www.analyticssteps.com/blogs/6-applications-augmented-reality-healthcare
\linebreak
IV.	https://imaginovation.net/blog/ar-in-healthcare-use-cases/


\end{document}
